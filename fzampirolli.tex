\documentclass[10pt,brazil,a4paper]{exam}
\usepackage[utf8]{inputenc}
%\usepackage[portuguese]{babel}
\usepackage[brazilian]{babel}
\usepackage[table,xcdraw]{xcolor}

%sudo apt install linuxbrew-wrapper
%sudo apt install texlive-extra-utils
%sudo apt install texlive-font-utils

\usepackage[pdftex]{graphicx}
\usepackage[dvips]{graphicx}
\usepackage{epstopdf}
\graphicspath{{./tmp/}}
\DeclareGraphicsExtensions{.pdf,.jpeg,.png,.eps}
\usepackage[outdir=./]{epstopdf}

\usepackage{hyperref} % include fig um http

\usepackage{listings} % para códigos

%\usepackage[top=.4in, bottom=.1in, left=1in, right=.4in]{geometry}
\textheight=6in
\usepackage{tikz}
\usepackage{enumitem}
%\usepackage[shortlabels]{enumerate}
\usepackage{multirow}
\usepackage{amsmath}
\usepackage{changepage,ifthen}
\usepackage{verbatim}
\usepackage{tabularx}
\usepackage{amsfonts}
\usepackage{multicol}

%\usepackage[T1]{fontenc}
%\usepackage{times}         
%\usepackage{multido}  % border
%\usepackage{pst-barcode}

\setlength{\textwidth}{185mm}
\setlength{\oddsidemargin}{-0.5in}
\setlength{\evensidemargin}{0in}
\setlength{\columnsep}{8mm}
\setlength{\topmargin}{-28mm}
\setlength{\textheight}{282mm}
\setlength{\itemsep}{0in}

\newcommand*\varhrulefill[1][0.4pt]{\leavevmode\leaders\hrule height#1\hfill\kern0pt}
\def\drawLines#1{{\color{lightgray}\foreach \x in {1,...,#1}{\par\vspace{2mm}\noindent\hrulefill}}}       
% cyan
\newcommand{\qid}[1]{\textcolor{white}{{\tiny #1}\hspace{0mm}}} %question identification (invisible text #1)
\usepackage[table,xcdraw]{xcolor}

#\renewcommand{\thefootnote}{\color{white}{\faGears}}
\renewcommand{\thefootnote}{\fnsymbol{footnote}}

\definecolor{bubbles}{rgb}{0.71, 1.0, 1.0}

\begin{document}
\pagestyle{empty}

\pagenumbering{gobble}
%\newgeometry{bottom=0.1cm}
\noindent\Huge{webMCTest}\normalsize\vspace{5mm}\\
\noindent\textbf{Topic:} template-equação-paramétrica\\
\noindent\textbf{Group:} \\
\noindent\textbf{Short Description:} q2\\
\noindent\textbf{Type:} QM\\
\noindent\textbf{Difficulty:} 1\\
\noindent\textbf{Bloom taxonomy:} remember\\
\noindent\textbf{Last update:} 2019-01-30\\
\noindent\textbf{Who created:} fzampirolli@ufabc.edu.br\\

\hspace{-15mm}{\small {\color{green}\#070}} \hspace{-1mm} 1. % texto da questão
O(a) \_\_\_\_\_\_ em Estrutura de Dados é também conhecido (a) como array Uni-dimensional. Assinale a alternativa que complete a lacuna.

% necessario para gerar casos de testes no moodle ==> mude verbatim <==> comment apos finalizar a questao

\begin{verbatim}
{"input": [["Uni", "lac-I"], ["entrada2"]], "output": [["Vetor", "lac-O"], ["saida2"]]}
\end{verbatim}

\n\n

\vspace{2mm}\begin{oneparchoices}
\choice \hspace{-2.0mm}{\tiny{\color{red}*2}}Pilha usando Ponteiro
\choice \hspace{-2.0mm}{\tiny{\color{red}*4}}Grafo
\choice \hspace{-2.0mm}{\tiny{\color{blue}\#0}}Vetor
\choice \hspace{-2.0mm}{\tiny{\color{red}*3}}Árvore
\choice \hspace{-2.0mm}{\tiny{\color{red}*1}}Matriz
\end{oneparchoices}\vspace{0mm}
\end{document}